\documentclass{beamer}
\usepackage[utf8]{inputenc}
\usepackage[T1]{fontenc}
\usepackage[portuges,brazilian]{babel}
\usepackage{graphicx}
\usepackage[sfdefault]{quattrocento}
\usepackage{multicol}
\usepackage{hyperref}
\usetheme[showheader,blue,colorblocks]{Verona}
\title{JPEG 2000}
\subtitle{Sistemas Multimídia}
\author[Yuri, Joel]{Yuri Oliveira \and Joel Rocha}
\institute[IFCE]{Instituto Federal de Ciência, Arte e Tecnologia}
\date{Dezembro, 2015}
\logo{\includegraphics[width=0.8cm]{figure/logo.jpg}}
\begin{document}
\begin{frame}
\titlepage
\end{frame}
\begin{frame}{\contentsname}
\tableofcontents
\end{frame}
\section{Introdução}
\begin{frame}{Introdução}
   \begin{itemize}
      \item Comprimir é necessário!
      \item Fotografias, páginas WEB, exames médicos.
      \item Facilitar o armazenamento e transmissão.
      \item Permite usar imagens menores com mesmo efeito.
      \item Considere uma imagem...
         \begin{itemize}
            \item Tamanho: 3 in x 4 in (7,62 cm x 10,16 cm)
            \item Resolução: 500 dpi
            \item Pixel: 3 bytes
            \item 3.000.000 pixels = 9.000.000 bytes = \textbf{9 MB} (7 disquetes)
         \end{itemize}
   \end{itemize}
\end{frame}
\subsection{Compressão de dados}
\begin{frame}{Compressão de dados}
   \begin{block}{\textit{Lossless}}
      \begin{itemize}
         \item Sem perda de dados.
         \item Maior processamento.
         \item Permite reconstrução.
         \item Desenhos técnicos, textos, quadrinhos, mapas.
         \item \emph{Run-lengh encoding} (Huffman, LZW).
         \item PNG, GIF, TIFF.
      \end{itemize}
   \end{block}
\end{frame}
\begin{frame}{Compressão de dados}
   \begin{block}{\textit{Lossy}}
      \begin{itemize}
         \item Com perda de dados.
         \item Maior compressão.
         \item Não permite reconstrução.
         \item Fotografia em geral.
         \item Descarta o que é imperceptível.
         \item JPEG, PGF, ICER.
      \end{itemize}
   \end{block}
\end{frame}
\subsection{JPEG 1992}
\begin{frame}{JPEG 1992}
   Método de compressão de imagens com perda de dados que usa a transformada discreta do cosseno (DCT) e obtém o resultado assumindo que altas frequências não são percebidas.
\end{frame}

\section{Melhorias}
\subsection{Compressão}
\subsection{Escalabilidade}
\subsection{Editabilidade}
\section{Implementação}
\section{Análise dos resultados}
\subsection{Exemplos}
\subsection{Comparações}
\section{Referências}
\begin{frame}{Referências}
   \begin{itemize}
      \item \href{http://stefan.winklerbros.net/Publications/adip2004.pdf}{EBRAHIMI, CHAMIK, WINKLER. JPEG vs. JPEG2000: An Objective Comparison of Image Encoding Quality}
      \item \href{http://faculty.gvsu.edu/aboufade/web/wavelets/student_work/EF/}{ELZINGA, FEENSTRA. JPEG 2000: The Next Compression Standard using wavelet technology}
      \item \href{http://www.mvnet.fi/index.php?osio=Tutkielmat&luokka=Yliopisto&sivu=Image_compression}{VESTOLA. A study about image compression}
   \end{itemize}
\end{frame}

\end{document}
