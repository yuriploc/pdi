\documentclass{beamer}
\usepackage[utf8]{inputenc}
\usepackage[T1]{fontenc}
\usepackage[portuges,brazilian]{babel}
\usepackage{graphicx}
\usepackage[sfdefault]{quattrocento}
\usepackage{multicol}
\usetheme[showheader,red,colorblocks]{Verona}

\title{JPEG 2000}
\subtitle{Sistemas Multimídia}
\author[Yuri, Joel]{Yuri Oliveira \and Joel Rocha}
\institute[IFCE]{Instituto Federal de Ciência, Arte e Tecnologia}
\date{Dezembro, 2015}
\logo{\includegraphics[width=0.8cm]{figure/logo.jpg}}
\begin{document}
\begin{frame}
\titlepage
\end{frame}
\begin{frame}{\contentsname}
\tableofcontents
\end{frame}
\section{Introdução}
\subsection{Lossy vs. lossless}
\begin{frame}{Lossy versus lossless}
   \begin{block}{Lossless}
      Sem perda de dados.
   \end{block}
   \begin{block}{Lossy}
      Com perda de dados.
   \end{block}
\end{frame}
\subsection{JPEG 1992}
\begin{frame}{JPEG 1992}
   Método de compressão de imagens com perda de dados que usa a transformada discreta do cosseno (DCT) e obtém o resultado assumindo que altas frequências não são percebidas.
\end{frame}

\section{Melhorias}
\subsection{Compressão}
\subsection{Escalabilidade}
\subsection{Editabilidade}
\section{Implementação}
\section{Análise dos resultados}
\subsection{Exemplos}
\subsection{Comparações}
\section{Referências}
\begin{frame}{Referências}

\end{frame}

\end{document}
